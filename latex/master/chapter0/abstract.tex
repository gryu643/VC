\chapter*{概要} %章番号をつけない
\thispagestyle{empty}

\thispagestyle{empty}
近年,バックホール回線に無線回線を用いる無線中継網が進展している.一般的に,無線バックホール回線
における伝搬路は静的とされており,伝搬路情報の変動は比較的穏やかとされている.このような
条件下では,送受信機の両方にフィルタを置く通信方式を使用することができ,その一つとして
ベクトル符号化(VC:Vector Coding)が存在する.

ベクトル符号化は,伝搬路行列とその整合フィルタ(MF:Matched Filter)からなる合成チャネル行列の
固有ベクトルを拡散符号とするCDMである.合成チャネル行列のエルミート性から,固有ベクトルが
お互いに直交する点と,受信側におけるMFによってパスダイバーシチ効果を得る点を特徴とする.
固有ベクトルの獲得に関しては,これまでチャネル推定を用いた方法が数多く検討されてきた.
しかし,チャネル推定や推定した伝搬路情報から固有値分解によって固有ベクトルを得るには,
非常に多くの演算が必要となり,実装のハードルが高い.

そこで,本論文では膨大な演算を極めて簡単な非線形処理と無線機間反復処理で代替する手法,
Ping-Pong-Loop(PPL)を提案している.PPLは,べき乗法を基本原理とした,チャネル情報の
推定を必要としない固有ベクトル獲得を特徴とする.PPLは非常に少ない演算で固有ベクトルを
獲得できる点で,チャネル推定を用いる方法に対して優位性を持つが,固有ベクトルの獲得に
ある程度の無線機間反復数を要するという課題がある.また,ベクトル符号化自体にも,
小さい固有値に対応するチャネルにおいて,伝送するデータの誤り率が高くなるという課題がある.
本論文では,これらの課題に対して3つの検討を行った.無線機間反復数の削減に関しては,
伝送特性の劣悪なチャネルに対応する固有ベクトルを獲得せず,PPLを打ち切るアルゴリズムを
提案した.また,送信電力分配を適用することによって誤り率の高いチャネルの救済を行った.
そして,最後に送信電力分配を,固有ベクトル足切りアルゴリズムに適用し,これら3つに関して
シミュレーションを用いて検証を行った.

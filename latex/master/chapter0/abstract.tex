\chapter*{概要} %章番号をつけない
\thispagestyle{empty}

\thispagestyle{empty}
近年のスマートフォンやタブレット等の通信端末の急激な増加により通信トラフィックの増加が問題と
なっている.
これに対してシステム容量の拡大が喫緊の課題とされているが,
通信の高速化を阻害する要因としてチャネル歪みがある.このチャネル歪みに対して強い耐性を
もつ直交周波数分割多重(OFDM)方式が無線LANやディジタルテレビ放送等に幅広く用いられている.
OFDM方式は,周波数領域において直交する狭帯域なサブキャリアを用いた多重伝送を特徴とする.
この狭帯域性がチャネル歪みに対する強いチャネル耐性を実現する.

このOFDM方式に対して,ある状況においてより優れた伝送特性を示す変調方式としてベクトル符号化(VC)
がある.VCは伝搬路行列とその整合フィルタからなる合成チャネル行列の固有ベクトルを拡散符号とする
符号分割多重(CDM)方式である.合成チャネル行列はエルミート行列となるため,その固有ベクトルは
互いに直交する.また,受信側に整合フィルタを置くことで遅延派の取りこぼしのない
パスダイバーシチを実現できる.拡散符号の獲得に関して,これまでチャネル推定を用いた方法が検討されてきたが,
チャネル推定や推定した情報から固有値分解する必要があるため,膨大な演算量が必要となる.

そこで,この演算量を簡単な非線形処理と無線機間反復で代替するPing-Pong-Loop(PPL)を提案している.
PPLはべき乗法を基本原理とし,任意ベクトルを無線機間で往復
させることで徐々に固有ベクトルに収束させる点を特徴とする.しかし,PPLによる固有ベクトルの獲得には
ある程度の無線機間反復数が必要になるという課題がある.また,VC自体にも小さい固有値に対応する
チャネルを使用する場合,誤り率が高くなるという課題がある.本論文では,この2つの課題に対して
チャネルの足切りアルゴリズムと送信電力制御というアプローチを試みた.また,送信電力制御を
足切りアルゴリズムに適用することによって,使用可能チャネルを最大化しつつ無線機間反復数を
削減できることを,シミュレーションを用いて検証した.
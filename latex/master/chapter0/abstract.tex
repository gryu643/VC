\chapter*{概要} %章番号をつけない

近年,スマートフォン等の情報端末の増加に伴い,バックホール回線に無線回線を用いる無線中継網の
需要が高まっている.一般的に無線会において高速伝送を行う場合,受信側において遅延波による
符号間干渉が生じるため,適応等価等の信号歪みへの対策が必要となる.無線中継網の中継回線は,
端末の移動を伴うアクセス回線に比べて静的であるため,送受信の双方において適応等価を行うことが
有効とされる.このような方式ベクトル符号化(\emph{VC:Vector Coding})がある.

ベクトル符号化は,伝搬路行列とその整合フィルタ(\emph{MF:Matched Filter})からなる合成チャネル行列の
固有ベクトルを拡散符号とするCDMである.合成チャネル行列のエルミート性から,固有ベクトルが
お互いに直交する点と,受信側におけるMFによってパスダイバーシチ効果を得る点を特徴とする.
合成チャネル行列の固有ベクトルの獲得に関しては,これまでチャネル推定を用いた方法が数多く検討されてきた.
しかし,チャネル推定や推定した伝搬路情報から固有値分解によって固有ベクトルを得るには,
非常に多くの演算が必要となり,実装のハードルが高い.

そこで,本論文では膨大な演算を極めて簡単な非線形処理と無線機間反復処理で代替する手法,
Ping-Pong-Loop(\emph{PPL})を提案している.PPLは,べき乗法を基本原理とした,チャネル情報の
推定を必要としない固有ベクトル獲得を特徴とする.PPLは非常に少ない演算で固有ベクトルを
獲得できる点で,チャネル推定を用いる方法に対して優位性を持つが,固有ベクトルの獲得に
ある程度の無線機間反復数を要するという課題がある.また,ベクトル符号化自体にも,
小さい固有値に対応するチャネルにおいて,伝送するデータの誤り率が高くなるという課題がある.
本論文では,これらの課題に対して以下の検討を行った.無線機間反復数の削減に関しては,
伝送特性の劣悪な固有チャネルに対応する固有ベクトルを獲得せずPPL処理を停止する,
固有ベクトル足切りアルゴリズムを提案した.
\thispagestyle{empty}
\clearpage
また,送信電力分配を適用することによって,各固有チャネルに対して適切に
送信電力を割り振り,誤り率の改善を行った.
そして,最後に送信電力分配を,固有ベクトル足切りアルゴリズムに適用し,これら3つに関して
シミュレーションを用いて検証を行った.
\thispagestyle{empty}
\clearpage
\addtocounter{page}{-2}
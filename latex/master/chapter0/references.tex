\begin{thebibliography}{99}
    \bibitem{yamo} Y.Yamo, H.Suda, N.Umeda, and N.Nakajima,
    ``Radio Access Network Design Concept for the Fourth Generation Mobile
    Communication System,'' in Proc VTC'00-Spring, 2000,pp.2285-2289.

    \bibitem{pabst} R.Pabst et al, 
    “Relay based deployment concepts for wireless and mobile broadband radio,” IEEE
    Com.Mag, pp.80-89, Sep.2004

    \bibitem{kasturia} S.Kasturia, J.T.Aslanis, and J.M.Cioffi, ``Vector coding for partial
    response channels'', IEEE Trans. On Information Theory, Vol36, No.4, July 1990.

    \bibitem{furukawa} 古川, ``符号の直交分離とパスダイバーシチを同時に表現する符号分割多重伝送'',
    信学技法, RCS2006-52, 2006年8月

    \bibitem{li} Z.Li and H.Furukawa, ``An enhanced vector coding and its application to delay
    spread MIMO channels'', Proc. IEEE APWCS2007, Aug, 2007.

    \bibitem{takeda} 竹田, 中川, ``Vector CodingとOFDMの性能比較に関する検討'', 信学技法,
    RCS2007-215, March 2008.

    \bibitem{takanashi} 高梨, 竹田, 安達, 中川, ``Vector Coding における送信ダイバーシチの適用効果に
    関する一検討'', 信学技法, RCS2008-49, 2008年7月.

    \bibitem{takano} 高野 , 安達 , 大槻 , 中川, ``チャネル推定誤差及びフィードバック遅延を考慮したVC伝送
    系における送受信重み選択に関する理論検討'', 信学技法, RCS2010-13, 2010年4月.

    \bibitem{takeda2} D.Takeda, Member and M.Nakagawa, ``Adaptive Modulation and Code channel
    Elimination for Vector Coding System'', IEICE Trans Commun. Vol.E92-B, No.5, May.2009.

    \bibitem{nagate} 長手厚史, 舛井淳祥, 藤井輝也, ``OFDM方式におけるチャネル推定方に関する一検討''
    , 信学技法, RCS2001-195, pp.85-91, January 2002.

    \bibitem{imamura} 今村大地, 原晋介, 森永規彦, ``パイロット信号を用いたOFDMにおける副搬送波再生法'',
     電子情報通信学会論文誌 B , Vlo.J82-B, No.3, pp.393-401, 1999年3月.

    \bibitem{takahata} 高畑文雄,``ディジタル無線通信入門'',培風館,2002.

    \bibitem{saitou} 斉藤洋一,``ディジタル無線通信の変復調'', 電子情報通信学会,1996.

    \bibitem{okumura} 奥村喜久,進士昌明,``移動通信の基礎'', 電子情報通信学会,コロナ社,1996.

    \bibitem{ibaragi} 茨木俊秀,福島雅夫,``最適化の手法'', 共立出版株式会社,1993.

    \bibitem{kanatani} 金谷健一,``線形代数セミナー 射影,特異値分解,一般逆行列'',共立出版株式会社,2018.
\end{thebibliography}

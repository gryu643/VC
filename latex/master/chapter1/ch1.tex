\chapter{序論}
 
近年,バックホール回線に無線回線を用いる無線中継網が進展している.\cite{yamo} \cite{pabst}
一般的に,無線バックホール回線において,伝搬路情報の変動は比較的緩やかとされており,このような
条件下で使用できる方式としてベクトル符号化(VC:Vector Coding)がある.伝搬路の変動が少ない場合,送受信機の両方に
フィルタを置くことが可能である.これにより,受信側において信号の最大比合成と変調信号の分離を
線形フィルタで実現することができる.
つまり,非線形フィルタと比較して簡単な処理で信号電力対雑音電力比の最大化と,
符号間干渉(ISI:Inter Symbol Interference)のないシンボル分離が可能となる.

VCは,伝搬路行列とその整合フィルタ(MF:Matched Filter)からなる合成チャネル行列の
固有ベクトルを拡散符号とするCDMである.合成チャネル行列がエルミート行列となることから,
拡散符号がお互いに直交性を示し,ISI無く変調信号を取り出すことができる.
また,受信側に設置するMFによって,遅延波の取りこぼしのないパスダイバーシチを実現することが
できる.\cite{furukawa} 無線バックホール回線等で現在広く用いられている変調方式として,
直交周波数分割多重(OFDM:Orthogonal Frequency Division Multiplexing)方式があるが,
ある条件下においてVCがOFDMよりも優れた伝送特性を示すことが知られている.
\cite{kasturia}\cite{furukawa,li,takeda,takanashi}

拡散符号の獲得については,これまでチャネル推定を用いる方法が検討されてきた.\cite{nagate} 
\cite{imamura} しかし,チャネル推定や推定した情報から固有値分解によって固有ベクトルを
獲得するには非常に多くの演算が必要になる.\cite{takano} \cite{takeda2}
これは,演算能力に劣る小型中継機器では大きな問題となる.
これに対して,膨大な演算を極めて簡単な非線形処理と無線機間反復処理に置き換える
Ping-Pong-Loop(PPL)を提案している.PPLはべき乗法を基本原理とし,任意ベクトルを2つの無線機間
で反復処理することで,徐々に合成チャネル行列の固有ベクトルに収束させる点を特徴とする.
これにより,少ない演算量での拡散符号獲得を可能にするが,PPLで拡散符号を得るには,現状ある程度の
無線機間反復をこなす必要がある.また,VC自体としても,小さい固有値に対応するチャネルで
伝送するデータの誤り率が高くなるという課題を有している.

本論文では,PPLに関する上記の2つの課題に関して,3種類の検討を行う.
まず1つ目として,一定の通信品質を満足しないチャネル分の
固有ベクトルを獲得せず,PPLの反復処理を停止する,固有ベクトル足切りアルゴリズムについて検討する.
これにより,不要な拡散符号を獲得する必要がないため,無線機間反復数を削減することができる.
次に2つ目として,各チャネルに対する送信電力分配によって,小さい固有値に対応するチャネルの
救済を行う.そして最後に,2つ目の送信電力分配を足切りアルゴリズムに組み込むことで,
一定の通信品質を満足するチャネル数を最大化しつつ,無線機間反復数の削減を行う.

本論文は以下のように構成される.第2章では提案手法の原理を支える数学的基礎概念とディジタル通信
に関する基礎概念を説明する.第3章では提案アルゴリズムの根幹となるPPL過程について説明を行い,
第4章にて各種検証の方針説明とシミュレーションによる検証を行う.最後に第5章において,本論文の
まとめと今後の展望について述べる.
\chapter{序論}

近年のスマートフォンやタブレット等の通信端末の急激な増加により通信トラフィックの増加が問題と
なっている.従来のマクロセル基地局では一つの基地局がカバーする端末数が多くなるため,端末一台
当たりの通信容量を十分に確保することが困難である.そこでスモールセル基地局と呼ばれるカバー
範囲の狭い基地局を導入する手法が進展している.スモールセル化とは,狭小セルを多数空間に配する
ことで,基地局あたりに接続される端末数を減らし,空間的な負荷分散を行うことで,単位面積当たりの
システム容量を拡大する方法である.面的なカバーエリアの拡充が何より優先されるモバイル通信網に
おいては,スモールセル化による通信エリアの整備に多くの基地局の敷設が必要となる.敷設工事費用
のうち,特に基地局と基幹網とを接続する有線回線,いわゆるバックホール回線の敷設費用が高い.
バックホール回線の敷設コストを抑制するため,当該回線を有線回線ではなく,無線回線によって代替
した無線バックホールシステムが注目されている.\cite{yamo} \cite{pabst}

無線によるバックホール回線において,システム容量を拡大する方法として通信の高速化が
挙げられる.通信速度の高速化は,システム容量の拡大の他にもブロードバンド通信を必要とする多様な
サービスの展開促進にも効果があり,これまでも数多くの研究がなされてきた.通信の高速化を妨げる
要因の一つとして,チャネル歪みがある.現在では,これに対して強い耐性を持つ直交周波数分割多重
(\emph{OFDM: Orthogonal Frequency Division Multiplexing})方式が無線LANやデジタルテレビ
放送などで幅広く用いられている.OFDM方式は,周波数領域において互いに直交する狭帯域サブキャリア
を用いた多重伝送を特徴とする.この狭帯域性によりチャネル歪みに対する強い耐性を有する.また,
各サブキャリアの直交性により,周波数スペクトルを重ねることで全体の信号帯域を縮小でき高い周波数
利用効率を示す.ガードインターバル(\emph{GI: Guard Interval})と呼ばれるガード区間を
設けることで,マルチパス伝送路における遅延派によるブロック間干渉(\emph{IBI: Inter Block
Interference})を防ぐことができる点もOFDM方式の利点の一つである.

さて,OFDMは正弦波を拡散符号とする符号分割多重方式(\emph{CDM: Code Division Multiplexing})
とも見ることができる.近年,特定の条件下でOFDMよりも高い伝送特性を有する多重方式として,ベクトル
符号化(\emph{VC: Vector Coding})が注目されている.\cite{kasturia} VCとは,伝搬路行列と整合フィルタ
(\emph{MF: Matched Filter})からなる合成チャネル行列の固有ベクトルを拡散符号とするCDMである.
合成チャネル行列のエルミート性より,拡散符号群は互いに直交性を示し,固有値は実数となる.
即ち,拡散符号群の直交性により,受信側における単純な逆拡散処理によって,変調された情報を符号間
干渉(\emph{ISI: Inter Symbol Interference})を起こすことなく取り出すことができる.
また,受信側の伝搬路行列に対応するMFを通過したVC波は,Rake受信機と同様に遅延波の取りこぼしのない
パスダイバーシチ効果を得ることができる.\cite{furukawa} パスダイバーシチ効果とは,受信側において逆拡散処理出力
である相関ピークを最大比合成することによって信号対雑音電力比を改善する手法である.OFDMでは,
遅延波の影響を取り除くことで,マルチパスフェージングによるチャネル歪みを打ち消しているため,
VCのようなパスダイバーシチ効果を得ることができない.以上の特徴により,VCはOFDMよりも優れた
伝送特性を示すことが確認されている.\cite{furukawa,li,takeda,takanashi} しかしながら,VCを実現するためには高精度に獲得された
拡散符号が必須となる.拡散符号の獲得には正確なチャネル推定と固有値分解
が必要となり,実用化へのハードルは高い.\cite{takano} \cite{takeda2} これに対し,OFDMの変復調は送受信機側に簡単なIDFT/DFT機構を
設ける,いわゆるワンタップ等価により実現され,VCに比べ実装のハードルが低い.以上のように,VCの
実現には演算コストを削減し,実装上のハードルを下げるための何らかの方策が必要と言える.

VCの実装ハードルを高めている最大の原因は,伝搬路情報の正確な推定を前提とする合成チャネル行列の
固有ベクトル獲得にある.伝搬路の推定が正確であれば,固有ベクトル自体はQR法等の固有値分解により
求めることができる.チャネル推定については,これまで非常に多くの先行研究が見られる.\cite{nagate} 
\cite{imamura} その多くが
パイロット信号(既知信号)を用いて,受信側で受信信号との自己相関を計算する方法をとっている.
チャネル推定を用いた従来のVCでは,推定したチャネル情報から合成チャネル行列を計算し,固有値分解に
反復解法を適用する必要があるため計算量が膨大になるという問題点がある.これは,演算能力の点で劣る
小型の中継機器等において大きな問題となる.また,チャネル推定と固有値分解の両方が受信側
にて行われるが,VCでは送信側においても拡散符号が必要とされる.即ち,受信側で獲得した
拡散符号またはチャネル情報を送信側に送信する必要があるという問題もある.

本論文では,チャネル推定情報から得られる合成チャネル行列の固有値分解による拡散符号獲得に必要と
される膨大な演算を,簡単な非線形処理を施した送受信機間往復処理に置き換える手法,Ping-Pong-Loop
(PPL)を提案している.PPLではべき乗法を基本原理としており,1対の送受信機間で任意ベクトル群を
互いに送受信しつつ両機において極めて単純な非線形処理を施すことで,徐々に固有ベクトルに収束させる
ことが可能であることを示している.即ち,従来のVCで必要とされたチャネル推定,固有値分解といった
演算を比較的単純な反復処理によって置き換えることで演算量の削減を実現している.
しかしながら,現在のPPLアルゴリズムでは固有ベクトルの収束までにある程度の反復回数を必要とする.
これはPPLの基本原理に起因するものであり,べき乗法を使用する以上不可避である.
べき乗法は,合成チャネル行列の固有値を降順に並べたとき第1固有値と第2固有値の比が大きいほど
固有ベクトルの収束が速くなり,逆に小さいほど収束が遅くなるという特徴がある.
しかし,一般的な伝搬路行列では,合成チャネル行列の第1固有値と第2固有値は極端に大きな比は持たない.
そこで,PPLにおける反復処理において,反復回数の削減手法としてBER(Bit Error Rate)理論値を指標
とした獲得固有ベクトル足切りアルゴリズムを提案する.
PPLは最も大きい固有値に対応する固有ベクトルから順番に逐次的に収束し,かつ小さい固有値
に対応する固有ベクトルほど獲得精度が悪化するという特徴を持つ.
また,合成チャネル行列の各固有値は各チャネルの伝搬路利得に対応するため,固有値が小さいチャネル
程受信信号が小さくなり誤り率が高くなる.即ち,既に反復処理によって収束した固有ベクトルを使用
して伝送を行った際のBER理論値を逐次的に計算し,当該理論値が一定の品質基準に満たない場合それ以降の
処理を行わないことで余計な反復回数を削減することができる.これにより,実質的な情報伝送量を失わずに
PPLのプロセスを改善できる.また,当該アルゴリズムに送信電力制御を適用することで,時間当たりの
伝送可能データ量を最大化することが可能である.

本論文は以下のように構成される.第2章では提案手法の原理を支える数学的基礎概念とディジタル通信
に関する基礎概念を説明する.第3章では提案アルゴリズムの根幹となるPPL過程について説明を行い,
第4章にて各種検証の方針説明とシミュレーションによる検証を行う.最後に第6章において,本論文の
まとめと今後の展望について述べる.
\chapter{序論}

近年のスマートフォンやタブレット等の通信端末の急激な増加により通信トラフィックの増加が問題と
なっている。従来のマクロセル基地局では一つの基地局がカバーする端末数が多くなるため、端末一台
当たりの通信容量を十分に確保することが困難である。そこでスモールセル基地局と呼ばれるカバー
範囲の狭い基地局を導入する手法が進展している。スモールセル化とは、狭小セルを多数空間に配する
ことで、基地局あたりに接続される端末数を減らし、空間的な負荷分散を行うことで、単位面積当たりの
システム容量を拡大する方法である。面的なカバーエリアの拡充が何より優先されるモバイル通信網に
おいては、スモールセル化による通信エリアの整備に多くの基地局の敷設が必要となる。敷設工事費用
のうち、特に基地局と基幹網とを接続する有線回線、いわゆるバックホール回線の敷設費用が高い。
バックホール回線の敷設コストを抑制するため、当該回線を有線回線ではなく、無線回線によって代替
した無線バックホールシステムが注目されている。

無線によるバックホール回線において、システム容量を拡大する方法として通信の高速化が
挙げられる。通信速度の高速化は、システム容量の拡大の他にもブロードバンド通信を必要とする多様な
サービスの展開促進にも効果があり、これまでも数多くの研究がなされてきた。通信の高速化を妨げる
要因の一つとして、チャネル歪みがある。現在では、これに対して強い耐性を持つ直交周波数分割多重
(\emph{OFDM: Orthogonal Frequency Division Multiplexing})方式が無線LANやデジタルテレビ
放送などで幅広く用いられている。OFDM方式は、周波数領域において互いに直交する狭帯域サブキャリア
を用いた多重伝送を特徴とする。この狭帯域性によりチャネル歪みに対する強い耐性を有する。また、
各サブキャリアの直交性により、周波数スペクトルを重ねることで全体の信号帯域を軽減でき高い周波数
利用効率を示す。ガードインターバル(\emph{GI: Guard Interval})と呼ばれるガード区間を
設けることで、マルチパス伝送路における遅延派によるブロック間干渉(\emph{IBI: Inter Block
Interference})を防ぐことができる点もOFDM方式の利点の一つである。

さて、
\chapter{序論}
 
近年,スマートフォンやタブレット等の通信端末の急激な増加に伴い,大容量通信への期待が高まっている.
これを実現する有望な方策は小型の無線基地局を面的に多数設置し,面的周波数利用効率を高めること
である.しかしながら,無線基地局を多数設置するには基幹回線に接続するための有線バックホール回線を
要するため,その敷設コストの増大が問題となる.
そのため,バックホールとして無線中継網を用いる手法が検討されている~\cite{yamo} \cite{pabst}.
 ところで,一般的に,無線回線において高速伝送を行う場合,受信信号には遅延波による符号間干渉
 (信号歪み)が発生するため,適応等化等の信号歪みのへ対策技術が必須となる~\cite{akaiwa}.
 無線中継網の中継回線は端末の移動を伴うアクセス回線と比べて比較的静的であるため,
 その中継回線品質を高めるには送信機と受信機の双方で適応等化を行うことが有効である.

 線形フィルタを用いて送受信等化を実現する技術として,ベクトル符号化(\emph{VC:Vector Coding})~\cite{kasturia}が
知られている.ベクトル符号化は伝搬路相関行列の固有ベクトルを重み係数とする線形フィルタを
送受信で用いる方式であり,非線形の信号処理を行うことなく,信号電力対雑音電力比の最大化と
無歪み伝送(符号間干渉無し)を可能とする.また,無線LANや携帯電話システム等で広く採用されている
直交周波数分割多重(\emph{OFDM: Orthogonal Frequency Division Multiplexing})方式と異なり,
誤り訂正符号化を行うことなくパスダイバーシチ効果を得ることができるため~\cite{furukawa},
無線中継網等の固定回線通信においてはベクトル符号化がOFDMよりも優れた伝送特性を示すことが
期待できる.~\cite{furukawa,li,takeda,takanashi}しかしながら,ベクトル符号化を行うには送受信機双方でチャネル情報が必要となるため[**],
チャネル推定とその固有値分解に非常に多くの演算が必要になる~\cite{takano} \cite{takeda2}.
これは,演算能力が限られた小型中継機を用いる場合,大きな制約となる.従来,べき乗法~\cite{strang}の原理に
基づき,固有値分解の演算を簡易な非線形処理と無線機間反復処理に置き換えるPing-Pong-Loop(PPL)が
提案されている.PPLは,任意ベクトルを2つの無線機間で反復処理することで,
徐々に合成チャネル行列の固有ベクトルに収束させる点を特徴とする.
しかしながら,PPLを用いることで送受信機の個々の演算量は軽減されるものの,
反復演算の収束値を得るために,特に固有値が小さい固有ベクトルを求めるために多大な無線機間反復を
こなす必要があり,解決すべき課題となる.
また,小さい固有値に対応する固有チャネルでは,チャネル利得が固有値の大きさに応じて小さくなることから,
比較的誤りやすくなるという課題がある.

本論文では,PPLにおける上記の課題を解決するために,以下の検討を行う.
第1に,獲得すべき固有ベクトル数を考慮したPPLにおける反復停止規範を提案する.
通信品質が基準以下である固有ベクトルに関してPPLの反復処理を停止することで,
獲得すべき固有ベクトルの足きりを行うものである.PPLアルゴリズムはべき乗法に基づくため,
固有値が小さい場合ほど収束が遅い. 提案方式では,固有値とBERの理論的な関係からPPLの反復毎に
BERを求め,それが所要値以下となる場合に演算を停止する.
それにより,不要な固有ベクトルを獲得する必要がないため,無線機間反復数を削減することができる.
第2に,PPLのための送信電力分配法を提案する.BER最少化規範に基づき送信電力を
各固有チャネルに最適分配することで利得の小さい固有チャネルの誤り率の改善を行う.
最後に,送信電力分配と足切りアルゴリズムを併用することで,反復回数を抑えながら,
一定の通信品質を満足するチャネル数の最大化を行う.

 本論文は以下のように構成される.第2章では提案手法の原理を支える数学的基礎概念と
ディジタル通信に関する基礎概念を説明する.第3章では提案アルゴリズムの根幹となるPPL過程について
説明を行い, 第4章にて本論文における検討方式の説明とシミュレーションによる検証を行う.
最後に第5章において,本論文のまとめと今後の展望について述べる.

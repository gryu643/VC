\chapter{基本的概念の解説}
本章では、本論文を理解する上で必要となる基本概念についての説明を行う。VCを理解するため材料として
エルミート行列の性質やべき乗法について説明する。また、無線通信の理解に関してディジタル変復調、
同期検波等について説明を行う。

\section{数学的基礎知識}
VCを理解するための事前準備として数学的基礎知識の説明を行う。本章では行列の要素が複素数である
複素行列を扱っている。

\subsection{随伴行列(共役転置行列)}
ある複素行列$\bm{A}$において、全要素で共役をとった上でその行列を転置させたものを$\bm{A}$の随伴行列(共役転置行列)
といい、$\bm{A^H}$と表す。

(例)\quad
$
  \bm{A} = \left[
    \begin{array}{cc}
      1-i & 2-6i \\
      5+i & 3 \\
      0 & 1+9i
    \end{array}
  \right]
$とすると、
\vspace{1mm}
\begin{equation}
    \bm{A^H} = \left[
        \begin{array}{cc}
            1-i & 2-6i \\
            5+i & 3 \\
            0 & 1+9i
        \end{array}
    \right]^H 
    = \left[
        \begin{array}{ccc}
            1+i & 5-i & 0 \\
            2+6i & 3 & 1-9i \\
        \end{array}
    \right] \nonumber
\end{equation}

\subsection{複素ベクトルの内積}
要素数の等しい複素ベクトル$X,Y$の内積を、以下で定義する。
\begin{equation}
    (\bm{X},\bm{Y}) = \bm{X^HY}
\end{equation}

\subsection{任意ベクトルの表記}
M次元空間内の任意の列ベクトルを$\bm{x}$、M個の直交した固有ベクトルを$\bm{u_1},\bm{u_2},\ldots,\bm{u_M}$
とすると、$\bm{x}$はM個の固有ベクトル及び定数$c_i(i=0,1,\ldots)$を用いて、以下のように表すこと
ができる。
\begin{equation}
    \bm{x} = \sum_{i=1}^M c_i\bm{u_i}
\end{equation}

\section{エルミート行列}
ある複素行列$\bm{A}$において、$\bm{A}$の随伴行列$\bm{A^H}$が$\bm{A}$と等しいとき、$\bm{A}$を
エルミート行列と呼ぶ。

(例) \quad
$
  \bm{A} = \left[
    \begin{array}{cc}
      1 & 3-4i \\
      3+4i & 2
    \end{array}
  \right]
$とすると、
\vspace{1mm}
\begin{equation}
    \bm{A^H} = \left[
        \begin{array}{cc}
            1 & 3-4i \\
            3+4i & 2
        \end{array}
    \right]
    = \bm{A} \nonumber
\end{equation}
であり、$\bm{A^H}=\bm{A}$となることから$\bm{A}$はエルミート行列であると言える。

エルミート行列の主な性質として以下の4つが挙げられる。

\vspace{5mm}
\noindent\textbf{(性質1) \quad 任意複素ベクトルとの関係} \\
$\bm{A}$がエルミート行列であれば、任意の列ベクトル$\bm{x}$に対して、$\bm{x^HAx}$は実数になる。\\
\vspace{3mm}
(証明)

任意複素ベクトル
$
    \bm{x} = \left[
        \begin{array}{c}
            u \\
            v
        \end{array}
    \right]
$
、エルミート行列
$
    \bm{A} = \left[
        \begin{array}{cc}
            a & b \\
            b^* & c
        \end{array}
    \right]
$
とすると、
\begin{eqnarray}
    \bm{x^HAx} &=& 
    \left[
        \begin{array}{cc}
            u^* & v^*
        \end{array}
    \right]
    \left[
        \begin{array}{cc}
            a & b \\
            b^* & c
        \end{array}
    \right]
    \left[
        \begin{array}{c}
            u \\
            v
        \end{array}
    \right] \nonumber \\
    &=& auu^* + cvv^* + b^*uv^* + bu^*v \nonumber \\
    &=& a|u|^2+c|v|^2+b^*uv^*+bu^*v
\end{eqnarray}

\noindent(2.3)式の右辺第1項、第2項はともに実数である。第3項、第4項は互いに複素共役となっており、これらの
和は実数部分の2倍になる。よって、$\bm{x^HAx}$は実数であると言える。

\vspace{5mm}
\noindent\textbf{(性質2) \quad 全固有値が実数} \\
エルミート行列の固有値はすべて実数である。 \\
\vspace{3mm}
(証明) \\
エルミート行列$\bm{A}$の固有値を$\bm{\lambda}$、その$\bm{\lambda}$に対応する$\bm{0}$でない
固有ベクトルを$\bm{x}$とする。行列とその行列の固有値・固有ベクトルとの関係より、
\begin{equation}
    \bm{Ax} = \lambda\bm{x}
\end{equation}
(2.4)式の両辺に$\bm{x^H}$をかけると、
\begin{equation}
    \bm{x^HAx} = \lambda\bm{x^Hx}
\end{equation}
(2.5)式の左辺は(性質1)により実数である。加えて、$\bm{x}\neq\bm{0}$より、$\bm{x^Hx}=||x||^2$
は正の実数である。したがって、$\lambda$は実数でなければならない。

\vspace{5mm}
\noindent\textbf{(性質3) \quad 固有ベクトルの直交性} \\
エルミート行列の固有ベクトルは、他のあらゆる固有値の固有ベクトルと直交している。 \\
\vspace{3mm}
(証明) \\
ある$2\times2$以上の大きさを持ったエルミート行列$\bm{A}$の2つの異なる$\bm{0}$でない固有ベクトルを
$\bm{x}$,$\bm{y}$とし、$\bm{x}$,$\bm{y}$に対応する固有値を$\lambda,\mu(\lambda\neq\mu)$と
する。行列とその行列の固有値・固有ベクトルとの関係((2.4)式)より、
\begin{eqnarray}
    \bm{Ax} &=& \lambda\bm{x} \\
    \bm{Ay} &=& \mu\bm{y}
\end{eqnarray}
(2.6)式を共役転置すると、
\begin{equation}
    \bm{x^HA^H} = \lambda\bm{x^H} \hspace{10mm} (\because\lambda は実数より、\lambda=\lambda^*)
\end{equation}
$\bm{A}$はエルミート行列であることから$\bm{A=A^H}$。これを(2.8)式に代入して、
\begin{equation}
    \bm{x^HA} = \lambda\bm{x^H}
\end{equation}
(2.9)式の両辺に右から$\bm{y}$をかけると、
\begin{equation}
    \bm{x^HAy} = \lambda\bm{x^Hy}
\end{equation}
一方、(2.7)式の両辺に左から$\bm{x^H}$をかけると、
\begin{equation}
    \bm{x^HAy} = \mu\bm{x^Hy}
\end{equation}
これら(2.10)式、(2.11)式より、
\begin{equation}
    \lambda\bm{x^Hy} = \mu\bm{x^Hy}
\end{equation}
$\lambda\neq\mu$であることから、$\bm{x^Hy}=\bm{0}$でなければならない。$\bm{x^Hy}$は
複素列ベクトル$\bm{x},\bm{y}$の内積を表しており((2.1)式)、それが$\bm{0}$であるということは
$\bm{x}$と$\bm{y}$は直交関係にある。

\vspace{5mm}
\noindent\textbf{(性質4) \quad スペクトル定理} \\
エルミート行列$\bm{A}$はその固有ベクトル群行列$\bm{U}$、固有値対角行列$\bm{D}$を用いて、
次のように分解できる。 \\
\begin{eqnarray}
    \bm{A} &=& \bm{UDU^H} = c_1\lambda_1\bm{u_1u_1^H}+c_2\lambda_2\bm{u_2u_2^H}+\ldots+c_n\lambda_n\bm{u_nu_n^H} \nonumber \\
    &=& \sum_i c_i\lambda_i\bm{u_iu_i^H}
\end{eqnarray}
\vspace{3mm}
(証明) \\
あるエルミート行列$\bm{A}$の固有値・固有ベクトルをそれぞれ$\lambda_i,\bm{u_i}$とする。
$\bm{u_i}$は正規直交基底であるから、任意ベクトル$\bm{x}$は(2.2)式のように分解される。
$\bm{x}$に左から$\bm{A}$をかけると、
\begin{equation}
    \bm{Ax} = \sum_i c_i\bm{Au_i} = \sum_i c_i\bm{\lambda u_i}
\end{equation}
が得られる。一方、(2.13)式の右辺に$\bm{x}$をかけると、
\begin{equation}
    \left[
        \sum_k \lambda_k\bm{u_ku_k^H}
    \right]\bm{x}
    = \left[
        \sum_k \lambda_k\bm{u_ku_k^H}
    \right]\sum_i c_i\bm{u_i}
    = \sum_{i,k} c_i\lambda_k\bm{u_ku_k^Hu_i}
    = \sum_i c_i\lambda_i\bm{u_i} \nonumber
\end{equation}
となり、これは(2.14)式と一致する。以上より、(2.13)式が成立することが証明された。

\section{べき乗法}

\section{VCの概要}
\subsection{チャネル行列}
\subsection{合成チャネル行列}
\subsection{VC計算過程}

\section{最適化問題の解法}
\subsection{KKT条件(Karush-Kuhn-Tucker condition)}
\subsection{ラグランジュの未定乗数法}

\section{ディジタル変復調}
\subsection{ディジタル変調}
\subsection{同期検波}
\subsection{信号の等価低域表現}

\section{フェージング伝搬路の影響}
\subsection{マルチパスフェージングの発生原理}
\subsection{レイリーフェージング}

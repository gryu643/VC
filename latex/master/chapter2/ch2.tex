\chapter{基本的概念の解説}

\section{数学的基礎知識}
\subsection{随伴行列(共役転置行列)}
\subsection{複素ベクトルの内積}
\subsection{任意ベクトルの表記}

\section{エルミート行列}

\section{べき乗法}

\section{VCの概要}
\subsection{チャネル行列}
\subsection{合成チャネル行列}
\subsection{VC計算過程}

\section{最適化問題の解法}
\subsection{KKT条件(Karush-Kuhn-Tucker condition)}
\subsection{ラグランジュの未定乗数法}

\section{ディジタル変復調}
\subsection{ディジタル変調}
\subsection{同期検波}
\subsection{信号の等価低域表現}

\section{フェージング伝搬路の影響}
\subsection{マルチパスフェージングの発生原理}
\subsection{レイリーフェージング}

\section{マルチキャリア変調}
\subsection{マルチキャリア伝送の原理}
\subsection{直交周波数分割多重(OFDM:Orthogonal Frequency Division Multiplexing)}
\subsubsection{ベースバンドOFDMの構成}
\subsubsection{ガードインターバル(Guard Interval)}
\subsection{符号分割多重(CDM:Code Division Multiplexing)}
\subsubsection{拡散系列による固有符号の直交}
\subsubsection{RAKE受信機}
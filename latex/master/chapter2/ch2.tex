\chapter{基本的概念の解説}
本章では、本論文を理解する上で必要となる基本概念についての説明を行う。VCを理解するため材料として
エルミート行列の性質やべき乗法について説明する。また、無線通信の理解に関してディジタル変復調、
同期検波等について説明を行う。

\section{数学的基礎知識}
VCを理解するための事前準備として数学的基礎知識の説明を行う。本章では行列の要素が複素数である
複素行列を扱っている。

\subsection{随伴行列(共役転置行列)}
ある複素行列$\bm{A}$において、全要素で共役をとった上でその行列を転置させたものを$\bm{A}$の随伴行列(共役転置行列)
といい、$\bm{A^H}$と表す。

(例)\quad
$
  \bm{A} = \left[
    \begin{array}{cc}
      1-i & 2-6i \\
      5+i & 3 \\
      0 & 1+9i
    \end{array}
  \right]
$とすると、
\vspace{1mm}
\begin{equation}
    \bm{A^H} = \left[
        \begin{array}{cc}
            1-i & 2-6i \\
            5+i & 3 \\
            0 & 1+9i
        \end{array}
    \right]^H 
    = \left[
        \begin{array}{ccc}
            1+i & 5-i & 0 \\
            2+6i & 3 & 1-9i \\
        \end{array}
    \right] \nonumber
\end{equation}

\subsection{複素ベクトルの内積}
要素数の等しい複素ベクトル$X,Y$の内積を、以下で定義する。
\begin{equation}
    (\bm{X},\bm{Y}) = \bm{X^HY}
\end{equation}

\subsection{任意ベクトルの表記}
M次元空間内の任意の列ベクトルを$\bm{x}$、M個の直交した固有ベクトルを$\bm{u_1},\bm{u_2},\ldots,\bm{u_M}$
とすると、$\bm{x}$はM個の固有ベクトル及び定数$c_i(i=0,1,\ldots)$を用いて、以下のように表すこと
ができる。
\begin{equation}
    \bm{x} = \sum_{i=1}^M c_i\bm{u_i}
\end{equation}

\section{エルミート行列}
ある複素行列$\bm{A}$において、$\bm{A}$の随伴行列$\bm{A^H}$が$\bm{A}$と等しいとき、$\bm{A}$を
エルミート行列と呼ぶ。

(例) \quad
$
  \bm{A} = \left[
    \begin{array}{cc}
      1 & 3-4i \\
      3+4i & 2
    \end{array}
  \right]
$とすると、
\vspace{1mm}
\begin{equation}
    \bm{A^H} = \left[
        \begin{array}{cc}
            1 & 3-4i \\
            3+4i & 2
        \end{array}
    \right]
    = \bm{A} \nonumber
\end{equation}
であり、$\bm{A^H}=\bm{A}$となることから$\bm{A}$はエルミート行列であると言える。

エルミート行列の主な性質として以下の4つが挙げられる。

\vspace{5mm}
\noindent\textbf{(性質1) \quad 任意複素ベクトルとの関係} \\
$\bm{A}$がエルミート行列であれば、任意の列ベクトル$\bm{x}$に対して、$\bm{x^HAx}$は実数になる。\\
\vspace{3mm}
(証明)

任意複素ベクトル
$
    \bm{x} = \left[
        \begin{array}{c}
            u \\
            v
        \end{array}
    \right]
$
、エルミート行列
$
    \bm{A} = \left[
        \begin{array}{cc}
            a & b \\
            b^* & c
        \end{array}
    \right]
$
とすると、
\begin{eqnarray}
    \bm{x^HAx} &=& 
    \left[
        \begin{array}{cc}
            u^* & v^*
        \end{array}
    \right]
    \left[
        \begin{array}{cc}
            a & b \\
            b^* & c
        \end{array}
    \right]
    \left[
        \begin{array}{c}
            u \\
            v
        \end{array}
    \right] \nonumber \\
    &=& auu^* + cvv^* + b^*uv^* + bu^*v \nonumber \\
    &=& a|u|^2+c|v|^2+b^*uv^*+bu^*v
\end{eqnarray}

\noindent(2.3)式の右辺第1項、第2項はともに実数である。第3項、第4項は互いに複素共役となっており、これらの
和は実数部分の2倍になる。よって、$\bm{x^HAx}$は実数であると言える。

\vspace{5mm}
\noindent\textbf{(性質2) \quad 全固有値が実数} \\
エルミート行列の固有値はすべて実数である。 \\
\vspace{3mm}
(証明) \\
エルミート行列$\bm{A}$の固有値を$\bm{\lambda}$、その$\bm{\lambda}$に対応する$\bm{0}$でない
固有ベクトルを$\bm{x}$とする。行列とその行列の固有値・固有ベクトルとの関係より、
\begin{equation}
    \bm{Ax} = \lambda\bm{x}
\end{equation}
(2.4)式の両辺に$\bm{x^H}$をかけると、
\begin{equation}
    \bm{x^HAx} = \lambda\bm{x^Hx}
\end{equation}
(2.5)式の左辺は(性質1)により実数である。加えて、$\bm{x}\neq\bm{0}$より、$\bm{x^Hx}=||x||^2$
は正の実数である。したがって、$\lambda$は実数でなければならない。

\vspace{5mm}
\noindent\textbf{(性質3) \quad 固有ベクトルの直交性} \\
エルミート行列の固有ベクトルは、他のあらゆる固有値の固有ベクトルと直交している。 \\
\vspace{3mm}
(証明) \\
ある$2\times2$以上の大きさを持ったエルミート行列$\bm{A}$の2つの異なる$\bm{0}$でない固有ベクトルを
$\bm{x}$,$\bm{y}$とし、$\bm{x}$,$\bm{y}$に対応する固有値を$\lambda,\mu(\lambda\neq\mu)$と
する。行列とその行列の固有値・固有ベクトルとの関係((2.4)式)より、
\begin{eqnarray}
    \bm{Ax} &=& \lambda\bm{x} \\
    \bm{Ay} &=& \mu\bm{y}
\end{eqnarray}
(2.6)式を共役転置すると、
\begin{equation}
    \bm{x^HA^H} = \lambda\bm{x^H} \hspace{10mm} (\because\lambda は実数より、\lambda=\lambda^*)
\end{equation}
$\bm{A}$はエルミート行列であることから$\bm{A=A^H}$。これを(2.8)式に代入して、
\begin{equation}
    \bm{x^HA} = \lambda\bm{x^H}
\end{equation}
(2.9)式の両辺に右から$\bm{y}$をかけると、
\begin{equation}
    \bm{x^HAy} = \lambda\bm{x^Hy}
\end{equation}
一方、(2.7)式の両辺に左から$\bm{x^H}$をかけると、
\begin{equation}
    \bm{x^HAy} = \mu\bm{x^Hy}
\end{equation}
これら(2.10)式、(2.11)式より、
\begin{equation}
    \lambda\bm{x^Hy} = \mu\bm{x^Hy}
\end{equation}
$\lambda\neq\mu$であることから、$\bm{x^Hy}=\bm{0}$でなければならない。$\bm{x^Hy}$は
複素列ベクトル$\bm{x},\bm{y}$の内積を表しており((2.1)式)、それが$\bm{0}$であるということは
$\bm{x}$と$\bm{y}$は直交関係にある。

\vspace{5mm}
\noindent\textbf{(性質4) \quad スペクトル定理} \\
エルミート行列$\bm{A}$はその固有ベクトル群行列$\bm{U}$、固有値対角行列$\bm{D}$を用いて、
次のように分解できる。 \\
\begin{eqnarray}
    \bm{A} &=& \bm{UDU^H} = c_1\lambda_1\bm{u_1u_1^H}+c_2\lambda_2\bm{u_2u_2^H}+\ldots+c_n\lambda_n\bm{u_nu_n^H} \nonumber \\
    &=& \sum_i c_i\lambda_i\bm{u_iu_i^H}
\end{eqnarray}
\vspace{3mm}
(証明) \\
あるエルミート行列$\bm{A}$の固有値・固有ベクトルをそれぞれ$\lambda_i,\bm{u_i}$とする。
$\bm{u_i}$は正規直交基底であるから、任意ベクトル$\bm{x}$は(2.2)式のように分解される。
$\bm{x}$に左から$\bm{A}$をかけると、
\begin{equation}
    \bm{Ax} = \sum_i c_i\bm{Au_i} = \sum_i c_i\bm{\lambda u_i}
\end{equation}
が得られる。一方、(2.13)式の右辺に$\bm{x}$をかけると、
\begin{equation}
    \left[
        \sum_k \lambda_k\bm{u_ku_k^H}
    \right]\bm{x}
    = \left[
        \sum_k \lambda_k\bm{u_ku_k^H}
    \right]\sum_i c_i\bm{u_i}
    = \sum_{i,k} c_i\lambda_k\bm{u_ku_k^Hu_i}
    = \sum_i c_i\lambda_i\bm{u_i} \nonumber
\end{equation}
となり、これは(2.14)式と一致する。以上より、(2.13)式が成立することが証明された。

\section{べき乗法}
対角行列の固有値・固有ベクトルを求める方法の一つにべき乗法がある。$\bm{A}$をM次のエルミート行列
とし、その固有値を$\lambda_1,\lambda_2,\ldots,\lambda_M$とする。$\lambda_i$に対応する
固有ベクトルを$\bm{u_i}$とすると、(2.4)式より、
\begin{equation}
    \bm{Au_i} = \lambda_i\bm{u_i} \nonumber
\end{equation}
となる。以下では、$|\lambda_1|>|\lambda_2|>\cdots>|\lambda_M|$の関係があるとする。
任意ベクトル$\bm{x}$に左から$\bm{A}$をn回乗算したものを$\bm{x_n}$とする。$\bm{x}$に左から
$\bm{A}$を1回乗算すると、
\begin{eqnarray}
    \bm{x_1} &=& \bm{Ax_0} = \bm{A}\sum_{i=1}^M c_i\bm{u_i} \hspace{10mm}(\because(2.2)式) \nonumber \\
    &=& \sum_{i=1}^M c_i\bm{Au_i} = \sum_{i=1}^M c_i\lambda_i\bm{u_i}
\end{eqnarray}
である。(2.15)式に繰り返し左から$\bm{A}$を1回乗算すると、
\begin{eqnarray}
    \bm{x_2} &=& \bm{Ax_1} = \bm{A^2x_0} = \sum_{i=1}^M c_i\lambda_i^2\bm{u_i} \nonumber \\
    \bm{x_3} &=& \bm{Ax_2} = \bm{A^3x_0} = \sum_{i=1}^M c_i\lambda_i^3\bm{u_i} \nonumber \\
    \vdots \nonumber
\end{eqnarray}
となり、$\bm{x_n}$は下記のようになる。
\begin{eqnarray}
    \bm{x_n} &=& \sum_{i=1}^M c_i\lambda_i^n\bm{u_i} \nonumber \\
    &=& \lambda_1^n
    \left(
        c_1+\bm{u_1}+\sum_{i=2}^M c_i
        \left(
            \frac{\lambda_i}{\lambda_1}
        \right)^n
        \bm{u_i}
    \right)
\end{eqnarray}
(2.16)式において、nが十分に大きい($\bm{A}$の乗算を十分繰り返す)場合、
\begin{equation}
    \left|\frac{\lambda_i}{\lambda_1}^n\right| << 1 \hspace{10mm} (i=2,3,\ldots,M)
\end{equation}
であるから、(2.16)式、(2.17)式より、
\begin{equation}
    \bm{x_n} = \lambda_1^nc_1\bm{u_1}
\end{equation}
が導かれる。(2.18)式は、任意ベクトル$\bm{x}$に繰り返し$\bm{A}$を乗算することで$\bm{x}$が
$\bm{A}$の第1(最大)固有ベクトル$\bm{u_1}$の定数倍に近づいていくことを意味している。
$\bm{x_n}$と$\bm{u_1}$は、大きさは違うが方向は同じであるため、$\bm{x_n}$を正規化することで
$\bm{u_1}$が得られる。正規化とはベクトルの各成分をベクトルの長さ(ノルム)で割ることにより、
ベクトルを単位ベクトルにすることである。

nを十分大きく取り$\bm{u_1}$を引き込んだ後、$\bm{u_1}$に$\bm{A}$を乗算することで$\bm{u_1}$に
対応する固有値$\lambda_1$は得られる。つまり、$\bm{u_1}$に$\bm{A}$を乗算した結果を$\bm{u_1^{\prime}}$
とすると、$\bm{u_1^{\prime}}$は$\bm{u_1}$の定数倍となり、この定数が固有値$\lambda_1$である。

エルミート行列$\bm{A}$はスペクトル定理により、(2.13)式のように分解できる。(2.13)式と一致する。以上より、
\begin{equation}
    \bm{A} - \lambda_1\bm{u_1}\bm{u_1^H} = \lambda_2\bm{u_2}\bm{u_2^H} + \cdots + \lambda_n\bm{u_n}\bm{u_n^H}
\end{equation}
であるから、(2.19)式の左辺$\bm{A} - \lambda_1\bm{u_1}\bm{u_1^H}$($\bm{A_2}$とする)の
第1固有値・固有ベクトルは、$\bm{A}$の第2固有値・固有ベクトルに当たる。すなわち、$\bm{A}$から
第1固有値・固有ベクトルを導出した手順を$\bm{A_2}$に適用することで、$\bm{A}$の第2固有値・
固有ベクトルが得られる。第2以降の固有値・固有ベクトルも、第2固有値・固有ベクトルと同様の手順で得られる。

\section{VCの概要}
最初にVCを説明するにあたって必要なチャネル行列の定義を行った上で、VCの概要について説明する。

\subsection{チャネル行列}
チャネルインパルス応答$h[n]$は以下で与えられている。
\begin{equation}
    h[n] = \left\{
        \begin{array}{ll}
            = 0 & (n<0 \quad or \quad n \geq K) \\
            \neq 0 & (n=0)
        \end{array}
    \right.
\end{equation}
伝送信号系列を$x[n]$とすると、受信信号系列$y[n]$は(2.20)式を用いて、次の畳み込み
演算で与えられる。
\begin{equation}
    y[n] = \sum_{m=0}^\infty h[n-m]x[m]
\end{equation}
送信変調ベクトルをN次の列ベクトル$\bm{X}(=x[0],x[1],\ldots,x[N-1])^T$、チャネル行列を
$\bm{H}$とすると、受信信号ベクトル$\bm{Y}$は(2.21)式を行列表記することで下記のように
与えられる。
\begin{equation}
    \bm{Y} = \bm{HX}
\end{equation}
ここで$\bm{Y}$は$M(=N+K-1)$次の列ベクトル$(=y[0],y[1],\ldots,y[M-1])^T$で、チャネル歪み
の影響でN次からM次になっている。Kはパス数である。$\bm{H}$はM行N列の行列で以下のように
表記される。
\begin{equation}
    \bm{H} = \left[
        \begin{array}{ccccc}
            h[0] & 0 & 0 & \ldots & 0 \\
            h[1] & h[0] & 0 & \ldots & 0 \\
            \vdots & h[1] & h[0] & \ldots & \vdots \\
            h[K-1] & \vdots & h[1] & \ldots & 0 \\
            0 & h[K-1] & \vdots & \ldots & h[0] \\
            0 & 0 & h[K-1] & \ldots & h[1] \\
            \vdots & \vdots & \vdots & \ldots & \vdots \\
            0 & 0 & 0 & \ldots & h[K-1]
        \end{array}
    \right]
\end{equation}

次にチャネル行列$\bm{H}$に対するMFが$\bm{H^H}$であることを示す。SNRを最大とするMFの
インパルス応答を$h_{MF}[n]$とする。$h[n]$と$h_{MF}[n]$を接続した連結システムのインパルス応答
$h^{\prime}[n]$は、次の畳み込み演算で与えられる。
\begin{equation}
    h^{\prime}[n] = \sum_{m=0}^{\infty} h[m]h_{MF}[n-m]
\end{equation}
(2.24)式において、$n=0$のときがSNR最大であり、その出力は、
\begin{equation}
    h^{\prime}[0] = \sum_{m=0}^{K-1} |h[m]|^2
\end{equation}
(2.25)式の畳み込み演算を行列表記すると、$\bm{H^{\prime}}=\bm{H_{MF}H}$となる。簡単のため、$N=3$、$K=2$
での$\bm{H}$を考え、$\bm{H_{MF}=\bm{H^H}}$として$\bm{H^{\prime}}$を計算、さらにこの$\bm{H^{\prime}}$
にインパルス$\bm{I}=(1 \quad 0 \quad 0)^T$を入力したときの出力を計算すると、
\begin{eqnarray}
    \bm{H^{\prime}}I &=& \left[
        \begin{array}{ccc}
            |h[0]|^2+|h[1]|^2 & h^*[1]h[0] & 0 \\
            h^*[0]h[1] & |h[0]|^2+|h[1]|^2 & h^*[1]h[0] \\
            0 & h^*[0]h[1] & |h[0]|^2+|h[1]|^2
        \end{array}
    \right]
    \left[
        \begin{array}{c}
            1 \\
            0 \\
            0
        \end{array} 
    \right] \nonumber \\
    &=& \left[
        \begin{array}{c}
            |h[0]|^2+|h[1]|^2 \\
            h^*[0]h[1] \\
            0
        \end{array}
    \right]
\end{eqnarray}
(2.25)式の1行1列に着目すると、
\begin{equation}
    |h[0]|^2+|h[1]|^2 = \sum_{m=0}^{2-1} |h[m]|^2 \nonumber
\end{equation}
であり、これは先に述べたSNR最大時の出力である。以上より、$\bm{H^H}$が$\bm{H}$に対するMFで
あることが分かる。

\subsection{合成チャネル行列}
\subsection{VC計算過程}

\section{最適化問題の解法}
\subsection{KKT条件(Karush-Kuhn-Tucker condition)}
\subsection{ラグランジュの未定乗数法}

\section{ディジタル変復調}
\subsection{ディジタル変調}
\subsection{同期検波}
\subsection{信号の等価低域表現}

\section{フェージング伝搬路の影響}
\subsection{マルチパスフェージングの発生原理}
\subsection{レイリーフェージング}

\chapter{まとめ}
VCは,伝搬路と伝搬路の整合フィルタからなる合成チャネル行列の固有ベクトルを拡散符号と
するCDMである.
VCの拡散符号は,合成チャネル行列のエルミート性から互いに直交性を示す.すなわち,
送受信機において拡散・逆拡散処理を施すことによって,シンボル間干渉(ISI)を
生じさせることなく各変調信号を取り出すことが可能である.
また,受信機側に伝搬路に対する整合フィルタを置くことで,遅延派の取りこぼしのない
パスダイバーシチを実現することができる.この点において,遅延派の影響を削除する
OFDMと比較して優れた伝送特性を示すとされている.
固有ベクトルの獲得については,これまで伝搬路推定による方法が検討されて
きた.従来の方法では,まず伝搬路推定によって正確な伝搬路情報を獲得する必要が
あることと,伝搬路行列から合成チャネル行列の計算を行った後に,固有値分解に
よって固有ベクトルを求める必要があり,膨大な計算量が必要となる.

チャネル推定を用いる手法に対して,伝搬路推定や固有値分解の演算に代わって,
ごく簡単な非線形処理と無線機間反復処理によって逐次的に固有ベクトルを獲得する,
PPL(Ping-Pong-Loop)という手法が存在する.
PPLは,演算能力の低い無線機でも簡単な無線機間反復処理によって固有ベクトルを
獲得することができる点をメリットとしているが,現状のアルゴリズムでは
多くの無線機間反復数を必要とするという課題があった.また,VC自体の課題として,
固有ベクトルに対応する固有値がチャネルの利得になることから,固有値が小さい
チャネルにおいて伝送する情報の誤り率が高くなってしまうという課題があった.

本論文ではこれらの問題の解決方法として以下の内容について検討を行った.
\begin{enumerate}
    \item 通信品質が基準以下である固有ベクトルに関してPPLの反復処理を停止する,固有ベクトル足切りアルゴリズム
    \item 各固有チャネルに適切に送信電力を分配して,BER特性の改善を図る送信電力分配手法
    \item 送信電力分配を固有ベクトル足切りアルゴリズムに適用
\end{enumerate}
そしてこれらの検討内容について,次のような検証結果を得ることができた.
\begin{enumerate}
    \item 無線機間反復数の大幅な削減
    \item BER特性の改善
    \item 無線機間反復数の削減に加え,通信品質を満足する固有チャネル数の最大化
\end{enumerate}
将来的なPPLの展望としては,現状では伝搬路の状態が頻繁に変化しない場合を想定しているが,
伝搬路の変化に合わせて既に獲得した拡散符号を追随して更新していく方式などに
検討の余地があると考える.また,PPLの原理上,無線機間の反復処理中に雑音の影響が
蓄積するという課題があるので,これについても検討を行いたい.

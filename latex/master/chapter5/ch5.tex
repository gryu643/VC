\chapter{まとめ}
VCは,伝搬路と伝搬路の整合フィルタからなる合成チャネル行列の固有ベクトルを拡散符号と
するCDMである.
VCの拡散符号は,合成チャネル行列のエルミート性から互いに直交性を示す.すなわち,
送受信機において拡散・逆拡散処理を施すことによって,シンボル間干渉(ISI)を
生じさせることなく各変調信号を取り出すことが可能である.
また,受信機側に伝搬路に対する整合フィルタを置くことで,遅延派の取りこぼしのない
パスダイバーシチを実現することができる.この点において,遅延派の影響を削除する
OFDMと比較して優れた伝送特性を示すとされている.
固有ベクトルの獲得については,これまで伝搬路推定による方法が検討されて
きた.従来の方法では,まず伝搬路推定によって正確な伝搬路情報を獲得する必要が
あることと,伝搬路行列から合成チャネル行列の計算を行った後に固有値分解に
よって固有ベクトル必要があり,膨大な計算量が必要となる.そこで,本論文では
伝搬路推定や固有値分解の演算に代わって,ごく簡単な非線形処理と無線機間反復処理
によって逐次的に固有ベクトルを獲得するPPL(Ping-Pong-Loop)を提案している.
PPLは,演算能力の低い無線機でも簡単な無線機間反復処理によって固有ベクトルを
獲得することができる点をメリットとしているが,現状のアルゴリズムでは
多くの無線機間反復数を必要とするという課題があった.また,VC自体の課題として,
固有ベクトルに対応する固有値がチャネルの利得になることから,固有値が小さい
チャネルにおいて伝送する情報の誤り率が高くなってしまうという課題があった.
この2点の課題に焦点を当て,本論文ではこれらの問題の解決方法として3つの内容に対して
検討を行った.まず,1つ目として,無線機間反復数を削減する目的で,
獲得固有ベクトル足切りアルゴリズムを提案した.シミュレーションの結果から,
$E_bN_0$が小さい領域において,当該アルゴリズム適用した場合で無線機間反復数を
削減することができた.また,2つ目に,利得の小さいチャネルを救済するため,
BER理論値を目的関数,各チャネルに割り当てる送信電力を変数とする最適化問題を
解くことによってBER理論値を最小化できることを示した.
最後に3つ目として,足切りアルゴリズムに送信電力制御を適用することによって,
最低限必要なBER基準を満足するチャネル数を最大化しつつ,固有ベクトルの獲得に
必要な無線機間反復数を大きく削減できることを示した.将来的なPPLの展望としては,
現状では伝搬路の状態が頻繁に変化しない場合を想定しているが,
伝搬路の変化に合わせて既に獲得した拡散符号を追随して更新していく方式などに
検討の余地があると考える.